\documentclass{beamer}
\usetheme{Boadilla}
\usepackage{minted}
% Include the biblatex package with APA style
\usepackage[
backend=biber,
style=apa,
sorting=ynt
]{biblatex}
% Add the biblatex source file
\addbibresource{PINN-Future-Study.bib}

% Set title, author etc.
\title{Studies on PINNs}
\subtitle{The PINN Module and Future Studies}
\author{A. Salih Taşdelen}
\institute{METU}
\date{\today}

\begin{document}

% Make title 
\titlepage

% Generate Table of Contents
\begin{frame}{Outline}
    \tableofcontents
\end{frame}

% Start the actual presentation
\section{The Progress}

\begin{frame}
\frametitle{The Progress}
\begin{itemize}
    \item Reproduced the laminar flow over a cylinder \cite{rao_physics-informed_2020}.
    \item Tried to understand the implementation and how it could be generalized. Since the source is in Tensorflow 1 \parencite{abadi_tensorflow_2016}, the user is the one who almost directly implements the computation graph.
    \item Read other papers that implements PINNs in TF2:
    \begin{itemize}
        \item \textbf{DeepXDE} by \cite{lu_deepxde_2021}. (Residual Based Adaptive Refinement)
        \item \textbf{Elvet} by \cite{araz_elvet_2021}. (Gradient Stack)
        \item \textbf{dNNSolve} by \cite{guidetti_dnnsolve_2021}. (Fourier Neural Nets)
        \item \textbf{IDRLnet} by \cite{peng_idrlnet_2021}. (Constraint Importance is proportional to its domain area)
        \item \textbf{PyDEns} by \cite{koryagin_pydens_2019}. (Deep Galerkin)
        \item \textbf{TensorDiffEq} by \cite{mcclenny_tensordiffeq_2021}. (\texttt{tf.gradients} instead of \texttt{tf.GradientTape})
    \end{itemize}
    
\end{itemize}    
\end{frame}

\section{Physics Informed Neural Networks}

\subsection{Problem in Mathematical Terms}

\begin{frame}{Problem in Mathematical Terms I}
\framesubtitle{The Domain}

\begin{itemize}
    \item Suppose we are given a problem in $D$ number of space time dimensions. Namely, we have time $t$, and $D-1$ number of spatial dimensions. Then, the domain can be represented as:
    \begin{itemize}
        \item $\mathrm{D} = [t_0, t_1] \times \Omega \subset \mathbb{R}^D$ where,
        \item $\Omega \subseteq \mathbb{R}^{D-1}$, is the spatial domain for which,
        \begin{itemize}
            \item $\vec{x} = \begin{bmatrix} x_1 & x_2 & \cdots & x_{D-1}\end{bmatrix}^T$ is an element of $\Omega$.
        \end{itemize}
    \item We can represent the elements of space-time domain as a column vector $\varphi = \begin{bmatrix} t & x_1 & x_2 & \cdots & x_{D-1}\end{bmatrix}^T \in \mathbb{R}^D$.
    \item With the above definitions, let us say that $\vec{u}(\varphi)$ is the desired solution of the problem.
    \item Also let $\hat{u}(\varphi)$ be the solution approximated by the Neural Network.
    \item We generalize the desired solution to be in any dimension, so that it could be the measure of various quantities like; velocity, pressure etc.
    \end{itemize}
\end{itemize}

\end{frame}

\begin{frame}{Problem in Mathematical Terms II}
\framesubtitle{Governing Equations - PDEs}    
Let us look at the governing equations as well:
\begin{itemize}
    \item Partial Differential Equations:
    \begin{itemize}
        \item Suppose the PDEs have the form $f_i(\varphi, \vec{u}(\varphi)) := 0$, $\varphi \in T_\Omega \subseteq \mathrm{D}.$
        \item If we have $n_\Omega$ number of PDEs their representation would be:
        \begin{align}
           \vec{f}(\varphi, \vec{u}(\varphi)) := \begin{bmatrix} f_1 & f_2 & \cdots & f_{n_\Omega}\end{bmatrix}^T = \vec{0} 
        \end{align}
        \item Let us denote the number of collocation points sampled for the PDEs as $N_\Omega$. Which is equivalent to say, $\left|T_\Omega\right|  = N_\Omega$.
    \end{itemize}
\end{itemize}
\end{frame}

\begin{frame}{Problem in Mathematical Terms III}
\framesubtitle{Governing Equations - ICs}

\begin{itemize}
    \item Initial Conditions:
    \begin{itemize}
        \item Suppose the ICs have the form $g_i(\varphi, \vec{u}(\varphi)) := 0$, $\varphi \in T_{0_i} \subseteq \mathrm{D}.$
        \item $\left|T_{0_i}\right| = N_{0_i}$, number of collocation points samples for the $i^{th}$ IC.
        \item For the initial condition we now that $t = t_0$, hence $T_{0_i} = \{t_0\} \times \mathrm{I}_i$.
        \item The spatial domain of IC is $\mathrm{I}_i \subseteq \Omega$.
        \item The spatial domains of initial conditions cannot coincide thus, $ \mathrm{I}_i \cap \mathrm{I}_j = \varnothing,\, i \neq j$.
        \item Let us define the set for all the sets for IC domains, $\Gamma_0 = \{T_{0_i}\}_{1\leq i \leq n_0}$.
        \item Since IC domains do not intersect we have the total number of IC collocation points as:
        \begin{align*}
            N_0 = \sum \limits_{i=1}^{n_0}\left|T_{0_i}\right|
        \end{align*}
    \end{itemize}
\end{itemize}
    
\end{frame}

\begin{frame}{Problem in Mathematical Terms IV}
\framesubtitle{Governing Equations - BCs}

\begin{itemize}
    \item Boundary Conditions:
    \begin{itemize}
        \item Suppose the BCs have the form $h_i(\varphi, \vec{u}(\varphi)) := 0$, $\varphi \in T_{0_i} \subseteq \mathrm{D}.$
        \item $\left|T_{{\partial\Omega}_i}\right| = N_{{\partial\Omega}_i}$, number of collocation points samples for the $i^{th}$ BC.
        \item For the boundary conditions, $T_{{\partial\Omega}_i} = \left[t_0,t_1\right] \times \mathrm{B}_i$.
        \item The spatial domain of BC is $\mathrm{B}_i \subseteq \partial\Omega$.
        \item The spatial domains of boundary conditions cannot coincide thus, $ \mathrm{B}_i \cap \mathrm{B}_j = \varnothing,\, i \neq j$.
        \item Let us define the set for all the sets for BC domains, $\Gamma_{\partial\Omega} = \{T_{{\partial\Omega}_i}\}_{1\leq i \leq n_{\partial\Omega}}$.
        \item Since BC domains do not intersect we have the total number of BC collocation points as:
        \begin{align*}
            N_{\partial\Omega} = \sum \limits_{i=1}^{n_{\partial\Omega}}\left|T_{{\partial\Omega}_i}\right|
        \end{align*}
    \end{itemize}
\end{itemize}
    
\end{frame}

\begin{frame}{Problem in Mathematical Terms V}
\framesubtitle{The Loss Functions}

Suppose we use MSE to compute the Neural Networks Loss. 

\begin{align}
    \mathrm{L}_0 := \mathrm{L}_0(\Gamma_0, \theta) &= \sum \limits_{T_{0_i} \in \Gamma_0} \frac{1}{\left|T_{0_i}\right|} \sum \limits_{\varphi \in T_{0_i}} {\left|\left| g_i(\varphi, \hat{u}(\varphi)) \right|\right|}_2^2 \\
    \mathrm{L}_{\partial\Omega} := \mathrm{L}_{\partial\Omega}(\Gamma_0, \theta) &= \sum \limits_{T_{{\partial\Omega}_i} \in \Gamma_{\partial\Omega}} \frac{1}{\left|T_{{\partial\Omega}_i}\right|} \sum \limits_{\varphi \in T_{{\partial\Omega}_i}} {\left|\left| h_i(\varphi, \hat{u}(\varphi)) \right|\right|}_2^2 \\
    \mathrm{L}_\Omega := \mathrm{L}_\Omega(T_\Omega, \theta) &=  \frac{1}{\left|T_{\Omega}\right|} \sum \limits_{\varphi \in T_{\Omega}} {\left|\left| f_i(\varphi, \hat{u}(\varphi)) \right|\right|}_2^2 \\
    L &= \alpha_0\mathrm{L}_0 + \alpha_{\partial\Omega}\mathrm{L}_{\partial\Omega} + \mathrm{L}_\Omega
\end{align}

\begin{block}{Residuals}
\small
Notice that the functions $f_i$, $g_i$,and $h_i$'s are the residuals that we are trying to minimize. In other words we want them to converge to $0$. Thus NNs loss can be written in terms of $\mathrm{L}(f_i(\varphi), \vec{0})$, $\mathrm{L}(g_i(\varphi), \vec{0})$, and $\mathrm{L}(h_i(\varphi), \vec{0})$.
\end{block}

\end{frame}

\subsection{Generalization}

\begin{frame}{Generalization}
Instead of separately handling PDEs, BCs, and ICs we can further generalize the residuals. We can think of each equation as a constraint, then each constraint has a domain and a residual. This way we will have only a single summation as a loss function.

\begin{align}
    \mathrm{L} := \mathrm{L}(\Gamma, \theta) &= \sum \limits_{T_i \in \Gamma} \frac{1}{\left|T_i\right|} \sum \limits_{\varphi \in T_i} C(g_i(\varphi, \hat{u}(\varphi)), \vec{0})
\end{align}

\begin{block}{Loss}
Here $C(y_{pred},y_{true})$ represents any loss function.
\end{block}

\end{frame}

\section{Python Module}

\subsection{How it Works}

\begin{frame}{How it Works I}

\begin{example}
\small
Let us consider the 1D Heat Conduction. The PDE, BCs, and IC are as follows:
\begin{align*}
\frac{\partial u}{\partial t} - 0.05\frac{\partial^2 u}{\partial x^2} = 0 \\
u(0,x) = sin(3\pi x) \\
\frac{\partial u}{\partial x}\bigg|_{\partial \Omega} = 0
\end{align*}
The analytical solution is: $u(t, x) = cos(3\pi x)e^{-.05(3\pi)^2t}$
\end{example}

\end{frame}

\begin{frame}{How it Works II}
\framesubtitle{Defining Constraints}
\small
\inputminted{python}{heat_constraints.py}
\end{frame}

\begin{frame}{How it Works III}
\framesubtitle{Defining Model and Training}
\small
\inputminted{python}{heat_training.py}
\end{frame}


\begin{frame}{How it Works IV}
\framesubtitle{Gradient Computation Algorithm}
\tiny
\inputminted{python}{gradient.py}
\end{frame}

\begin{frame}{How it Works V}
\framesubtitle{Training Algorithm}
\small
\inputminted{python}{training.py}
\end{frame}

\subsection{New Features to be Added}

\begin{frame}{New Features}

\begin{itemize}
    \item A seperate module for domain generation.
    \item Efficient gradient calculation.
    \item Ready to use constraints, Neumann BCs, Dirichlet BCs, etc.
    \item Numeric derivative computation.
\end{itemize}

\end{frame}


\section{Future Work}

\begin{frame}{Future Work}

\begin{itemize}
    \item Residual based adaptive refinement.
    \item Constraint weights proportional to its domain area.
    \item Fourier Networks in combination to standard networks.
    \item Dimensionless inputs and outputs.
\end{itemize}

\end{frame}



% Dump all the references at once
\begin{frame}[allowframebreaks]
\frametitle{References}
\tiny
\nocite{*}
\printbibliography
\end{frame}

\end{document}
